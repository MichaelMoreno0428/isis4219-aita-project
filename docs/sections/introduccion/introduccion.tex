\section{Introducción}

Las redes sociales han revolucionado la manera en que las personas interactúan, comparten opiniones y construyen comunidades. Reddit es de las redes sociales más particulares. Especialmente conocida por su estructura y sus foros especializados en los que los usuarios pueden debatir sobre cualquier tema, de forma anónima. Un aspecto interesante de Reddit son sus "subreddits", foros o comunidades centradas en un tema donde los usuarios pueden crear y compartir publicaciones, comentarlas y votar en ellas. 

Dentro de estas comunidades, existe r/AmITheA**hole (AITA), un espacio donde los usuarios comparten historias personales sobre situaciones complejas y piden a la comunidad que los juzgue o aconseje, respondiendo a la pregunta: "¿Soy el a**hole (el malo de la historia)?". Este tipo de interacción genera una fuente de datos o dataset único que puede servir de insumo para uno o varios modelos de aprendizaje automático o machine learning; los cuales pueden ayudar con el objetivo de analizar cómo se desarrollan las dinámicas sociales en línea, cómo se toma una decisión colectiva y si es posible automatizarla, y qué factores influyen en la aceptación o rechazo de una historia. 

Esta comunidad ha ganado una popularidad considerable; las publicaciones varían en complejidad, desde simples malentendidos hasta situaciones más serias que involucran relaciones familiares, parejas, amigos o compañeros de trabajo. Los participantes, al presentar sus historias, esperan obtener un veredicto, en la cual se tienen en cuenta los comentarios de la publicación, donde los participantes de la comunidad opinan utilizando un comentario con el voto específico, el cual puede ser YTA you’re the a**hole (eres es el malo), NTA not the a**hole (no eres el malo), ESH everyone sucks here (todos son malos) entre otros votos posibles, al igual que expresan sus opiniones respecto a la situación y los actores. Sin embargo, los votos a favor o en contra (upvotes o downvotes), no están ligados al veredicto, sino a la popularidad y qué tan interesante es el post.

Existen varios problemas que afectan la calidad del contenido y la carga de trabajo de los moderadores de la comunidad. A medida que la comunidad crece, los moderadores pueden verse sobrecargados con la gestión de las publicaciones, lo cual consume mucho tiempo, especialmente si se alcanza una cantidad significativa de publicaciones diarias. Esta saturación no solo afecta la efectividad de los moderadores, sino que también puede influir en la calidad de las interacciones y en la precisión de los veredictos que se emiten.

Esto abre la posibilidad para considerar una herramienta automatizada que apoye a los moderadores a poner el veredicto en la publicación. La inclusión de dicha herramienta sería en complemento a su labor actual, donde el análisis de publicaciones se puede realizar de manera más eficiente, permitiendo que los moderadores se enfoquen en aspectos más subjetivos y delicados, mientras que el sistema se encargaría de dar una sugerencia de acuerdo al contenido de la publicación.

Otro desafío significativo es la gestión de la interacción entre los moderadores y los usuarios. Si bien los moderadores deben ser imparciales, es posible que existan casos donde los prejuicios personales influyan en el veredicto de una publicación, lo cual podría generar conflictos dentro de la comunidad. Además, las reglas del subreddit pueden no ser siempre claras para todos los usuarios, lo que genera cierta ambigüedad sobre qué publicaciones son aceptables y cuáles no. La existencia de subcategorías, como WIBTA (Would I Be The A**hole), que plantea situaciones hipotéticas sobre el futuro, requiere una segmentación más clara entre los distintos tipos de publicaciones para asegurar una correcta categorización y evaluación.

Es por ello que se considera de interés investigar los temas que más frecuentemente influencian los veredictos, como las relaciones familiares o las disputas laborales, para poder identificar patrones en la toma de decisiones y en las dinámicas que subyacen a los votos. ¿Por qué algunas historias reciben más popularidad que otras? ¿Es el contenido de las publicaciones lo que determina su aceptación o el perfil del usuario que las comparte? ¿Qué tan relevantes son la longitud del texto y otros factores en la popularidad de un post? Estas preguntas requieren una investigación para entender cómo el algoritmo de Reddit y la interacción humana juegan un papel crucial en los resultados finales.

r/AmITheA**hole ofrece una oportunidad para comprender mejor las dinámicas sociales en línea, así como para investigar cómo se toman decisiones colectivas en comunidades virtuales. Este conjunto de datos es lo suficientemente grande y diverso como para permitir un análisis profundo de los factores que influyen en la popularidad de las publicaciones y en los veredictos que emiten los usuarios. La variedad de temas tratados en las publicaciones hace que este análisis sea aún más relevante, ya que abarca una gama amplia de situaciones personales y sociales.

Además, este análisis es un ejercicio académico interesante, ya que permite aplicar técnicas de procesamiento de lenguaje natural (NLP) y modelos de machine learning para examinar cómo los usuarios interactúan en este tipo de plataformas. Combinar ambos enfoques ofrece una oportunidad para probar la efectividad de los modelos de manera práctica, obteniendo resultados que puedan contribuir al aprendizaje sobre el funcionamiento de estos sistemas. Finalmente, este ejercicio también puede llegar a ser interesante en la evaluación de modelos extensos de lenguaje (LLMs), utilizando como prompt el post y preguntándole cuál veredicto daría, con el fin de probar programáticamente qué tan acertados son.

Este estudio tiene el potencial de mejorar la comprensión de los mecanismos que rigen las decisiones de los usuarios en Reddit y en otras plataformas de redes sociales, lo que puede abrir nuevas posibilidades para la automatización de tareas y la creación de herramientas más eficientes para la moderación de contenido. Al mismo tiempo, este tipo de investigación puede ser útil para desarrollar algoritmos que no solo analicen el contenido de las publicaciones, sino que también reconozcan los matices humanos que influyen en las decisiones de las personas.


\subsection{Objetivo General}



\subsection{Objetivos Específicos}

\begin{enumerate}
    \item I
\end{enumerate}